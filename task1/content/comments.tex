\section{The seven UN core human rights conventions}
In their engagement for an universal equality of all human beings,
isn't it contradictionary to declare a \emph{Convention on the Elimination
of All Forms of Disrimination Against Women}?
\cite[p. 18]{tfohr}

The second article mentiones the sex as an explicit attribute which is not
used for differentiation in terms of human rights:\emph{
``[...] entitled to all the rights and freedoms set forth in this Declaration,
without destinction of any kind, such ad race, colour, sex, language, religion,
political or other opinion, national or social origin, property, birth or
other status.''}
\cite[p. 14]{tfohr}

In the paragraph \emph{Protection for People with Special Needs} the author
describes the specific and explicit declaration for certain groups (the author
actually uses the terminus \emph{class of citizens}) as a necessity introduced
by the lack of equality or insufficient equality.
\cite[. 29]{tfohr}



In the case of children or disabled people and similar, I can totally agree
that these need specific declarations on their rights as human beeings. But
a differentiation on the level of human rights on adults introduced by their
sex is a statement towards inequality per definition and should be avoided
in my point of view. As long as there is a categorisation, there will always
be a difference in some way, even if it's only in the language used e.g. by
using different termini suggesting different conditions based on the sex.
And as soon as a differentiation is possible, people will differentiate
as they always did.

\section{The three core regional human rights conventions}
Similar to the previous section, the differentiation on human rights by
geographical locations contradicts the aspired universal validity and integrity
of human rights. So the question arises is if the specific declaration of the
human rights is a concept that is fit for the future development of the human
rights or if it's something we can consider as a necessary imperfection we have
to take until a universal declaration will cover all human beings around the
globe?
\cite[p. 18]{tfohr}

\section{Civil and political rights}
\subsection{Protection of liberty}
The paragraph \emph{Protection of liberty} declares the ``Prohibition of
imprisonment for debts''.
\cite[p. 20]{tfohr}

Isn't it a common practice among many nations to send people to prison for
unbalanced debts to governmental institutions e.g. fines?
\cite{busse}

\subsection{Freedoms}
The paragraph \emph{Freedoms} declares the ``Prohibition of arbitrary or
unlawful interference with one's privacy, family, home or correspondence''.
\cite[p. 20]{tfohr}

Are current developments towards permanent monitoring of all citizens without
specific reason or suspicion, thus arbitrary surveillance, in contradiction
with the human rights? In my point of view it is. Comparing an arbitrary
communication technology (phone, mail) with a face-to-face conversation, the
violation of one's privacy is obvious if we imagine someone coming constantly
between the two conversation participants noting each and every word. Or if we
imagine someone physically walking constantly with us, noting each and every
step we make. Most of the time these topics are discussed in the scope of
national levels withing the \emph{protection of data privacy}. Since the
massive interception of private data by governmental institutions is not only
limited to national monitoring but global monitoring beyond national borders,
will the human rights declaration give a chance for protection against such
practices when national law does not apply or it simply fails (like in the
NSA's aourveillance affairs)?
\cite{nsa}

\subsection{Freedom of marriage}
The ``Freedom of marriage'' is quite interesting because in nowadays
developments one could ask for a precise definition of this statement.
\cite[p. 20]{tfohr}

Is the terminus \emph{marriage} related to a general partnership or a legal
marriage as declared in most countries. If so, is the legal refuse of same
sex marriage violating the human rights or do they only declare the
partnership as such?

\section{The decline of national sovereignity}
The author declares in this section, that the human rights have become
part of the international law and implies consequences for individual states
and their relations to one another. ``National sovereignty is not only
restricted from above but also to a certain extent from below: state
sovereignty is circumscribed wherever victims of human rights violations
or their families take their fate into their own hands and bring their case
before an international body responsible for the protection of human rights.''
\cite[p. 19,21]{tfohr}

The international courts nevertheless seem to lose their acceptance
dramatically if they interfere with national interests. A current example is
the political movement in Switzerland which aims to declare some
rearrangement in hierarchy of laws in such manner that the international law
can be declared invalid by national law.
\cite{svp}

\section{The content of human rights}

\subsection{The first generation: Civil and political rights}
The author describes that the civil and political rights recognised by
international law include ``[\dots] freedoms and liberties such as
freedom of religion or of marriage [\dots]''.
\cite[p. 21]{tfohr}

What makes the marriage noteworthy in such a scope? Were marriages
illegal or otherwise complicated at the time? And why is it mentioned
just after the freedom of religion? Is there any bound to religious
aspects or is it just considered as a legal status of partnership in
general? 

\subsection{The second generation: Economics, social and cultural rights}
The author gives an example of an illegal act with ``the confiscation
of their houses without any compensation violates the right to housing
in the sense of the freedom to use one's place to live''.
\cite[p. 23]{tfohr}

Looking a the Israel-Palestinian conflict, this example could be
considered to be one of the most long-term and systematical violations
since the creation of the human rights movement. The government
creates new settlements on territory which is systematicall conquered
by forced migration of specific ethnical criteria. This strategy, which
is meanwhile going on for over half a century, results in no serious
international actions. So the question is, where can this example of
the author be applied to?

\subsection{Social rights}
``Right to adequate standard of living, including the right to adequate
food, clothing and housing''
\cite[p. 22]{tfohr}

Well this is a very special point if we consider the situation at
different parts of this world. The fact that many countries face
poverty issues even in highly developed parts of the world like in
northern Amerika or western Europe. Is it a crime, that in such
developed (or wealthy) countries people have to face heavy poverty
(like homeless people etc.)?

\section{Are human rights also binding on individuals and private bodies?}
``Anyone who commits genocide or is guilty of war crimes against humanity,
thereby commiting very serious human rights violations, may nowadays be
braought before and convicted by national or even international courts.''
\cite[p. 27]{tfohr}

I would like to say that I totally agree with this sentence and
appreciate it's meaning to our world but the fact how this is practiced
makes me more angry than happy. Considering the yugoslavian wars as a
prominent positive example of war crime tribunals, it's absolutely and
without any doubt useless to convict some soldiers and some of their
commanders. What does such a convictment change? Nothing at all is the only
answer. What does change something are reparations. National obligations
for their acts in war. The big issue in the yugoslavian wars is the fact,
that it's without a doubt accepted and takenn to be a so called
\emph{civil war}. With this definition, any obligation or responsibility
is taken away from all institutions execpt from some individulas. Any
other crime commited by individuals is also processed on national level.
The yugoslavian war tribunals seem to be one of the most prominent examples
but in my point of view it's not a positive one, because there is no step
to justice whatsoever.

\section{International criminal tribunals}
``How successful the international criminal courts will be is not yet
clear''
\cite[p. 33]{tfohr}

I can totally agree on this sentence. Taking a look on current actions and
movements on national politics can be considered alarming. With the example
of the USA, which always have been prominent in declining conventions, the
author even states ``the USA is even actively resisting this new institution
for the administration of criminal justice'' \cite[p. 33]{tfohr}. Looking at
those prominent examples one could ask if this is the kind of position
where switzerland is heading to with it's growing repression to international
law.

\section{Cross-cultural critique}
The author describes the meaning of the individual and the persons and the
difference it makes talking about those. He further explains the relativist
critique of the human rights suggesting a universal nature of human rights
and that a majority of the world ratified the human rights conventions thus
showing the great awareness and development. But he also states ``all
cultural traditions gratly value the dignity of the human person, and no
culture justifies arbitrary killing, genocide or brutal torture as values to
be defeated'' \cite[p. 36]{tfohr}.

I highly disagree with this sentence due several historical and sociological
facts. In all serious cases of human rights violations, like genocides or
otherwise systematical killing or torture, the victims were never considered
as a human person of the same worthiness, if they were even seen as human
beings at all. In every historical and current example of violations against
human rights, the crime was performed against victims which were considered
less human as the offender and all of those actions were highly supported by
the culture of that time. In some cases those cultural aspects are politically
suppressed and in others again they are not, resulting in human rights
violations in past decades and even today. To name some
examples\footnote{Wikipedia holds an article of ``genocied in history'',
which is shockingly long, where I face difficulties to find an example where
I can't see a cultural acceptance of the crime performed.} of this nature:

\begin{itemize}
\item suppression of the cathars' (1209 - 1220)
\item national socialist germany (till 1945)
\item Israel (since 1948)
\item yugoslavia (1991 - 1995)
\item Rwanda (1994)
\item ISIL (since 2014)
\end{itemize}

\section{Conclusion}
``Nevertheless, states and international organisations must have the
resolve to apply these principles effectively and take the necessary
action against those who violate them, regardless of political or
economic considerations.''
\cite[p. 37]{tfohr}

This seems to be an idealistic idea and the right thing to aim at but the
reality if far away from it. Especially the UN security council veto power
seem to ruin the idea on a regular basis by their own national interests.
In the history of the security council from 1946 to 2016 vetos were issued
on 236 occasions \cite{veto}. Looking at some examples of those vetos, it
is clear that national interests are considered and some other examples
are just questionable. The following list shows some notable examples
by the USA, which voted 79 resolutions.

\begin{itemize}
\item Condemning Israel for the death of hundreds of people in Syria and
Lebanon in air raids (USA, 1972).
\item Affirming the rights of the Palestinians and calling on Israel to
withdraw from the occupied territories (USA, 1973).
\item Condemning Israel for attacking Lebanese civilians (USA, 1976).
\item Condemning Israel for building settlements in the occupied territories
(USA, 1976).
\item Calling for self-determination for the Palestinians (USA, 1976).
\item Affirming the rights of the Palestinians (USA, 1976).
\item Condemning the attempts by South Africa to impose apartheid in Namibia
(USA, 1976).
\item Urging the permanent members (USA, USSR, UK, France, China) to
implement the United Nations decisions in the maintenance of international
peace and security (USA, 1978).
\item Criticizing the living conditions of the Palestinians (USA, 1978).
\item Calling for the return of all inhabitants expelled by Israel
(USA, 1979).
\item Demanding that Israel desists from human rights violations (USA, 1979).
\item Requesting a report on the living conditions of Palestinians in the
occupied Arab countries (USA, 1979).
\item Urging Israel to return displaced persons (USA, 1980).
\item Condemning the Israeli policy on the living conditions of the
Palestinian people (USA, 1980).
\item Condemning Israeli human rights practices in the occupied territories
(USA, 1980).
\item Urging negotiations on the prohibition of chemical and biological
weapons (USA, 1981).
\item Declaring that education, work, health, adequate nutrition, national
development, etc. are human rights (USA, 1981).
\item To discuss the issue of Palestinian refugees in the Gaza Strip
(USA, 1981).
\item Regarding Israeli human rights violations in the occupied territories
(USA, 1981).
\item For the ratification of the convention on the suppression and
punishment of apartheid (USA, 1982).
\item For a better development of international law (USA, 1982).
\item Support for the right to food (USA, 2008)
\end{itemize}

At the end the author states ``human rights are not a gift but a task for
all of us'' \cite[. 37]{tfohr} and I totally agree. To quote the last
word of the authors conclusion ``If people fail to take action on
behalf of their fellow human beings, it they lack sympathy for their
suffering and do not show soldarity with the victims of human rights
violations, if they do not cry out against oppression and disregard for
human dignity and if they do not persist in calling for more justice,
there can ultimately be no real peace in our world'' \cite[p. 37]{tfohr}.
